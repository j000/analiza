\documentclass{article}
% packages
% page geometry
\usepackage[
	a4paper,
	margin=1.5cm,
	includehead,
	includefoot
]{geometry}
% utf8 support
\usepackage[utf8]{inputenc}
% because...
\usepackage[T1]{fontenc}
% language support
\usepackage[polish]{babel}
% indent fist paragraph
\usepackage{indentfirst}
% better math
\usepackage{amsmath}
% math font
\usepackage{amsfonts}
% math stuff
\usepackage{mathtools}
\usepackage{amscd}
% proofs
\usepackage{amsthm}
\renewcommand{\qedsymbol}{\textsquare}

%different font
\usepackage{tgheros}
\renewcommand{\familydefault}{qhv}
% \usepackage{tgadventor}
% \renewcommand{\familydefault}{qag}

% for images
\usepackage{graphicx}
% for images
\usepackage{wrapfig}

% hyperlinks formatting
\usepackage[
	colorlinks,
	linkcolor=black,
	urlcolor=black,
	citecolor=black,
	plainpages=false,
	pdfpagelabels,
	unicode,
	backref
]{hyperref}
\makeatletter
\renewcommand{\Hy@numberline}[1]{#1. }
\makeatother

% for \LastPage
\usepackage{lastpage}

%custom headers and footers
\usepackage{fancyhdr}
\pagestyle{fancy}
\fancyhf{}
\renewcommand{\headrulewidth}{0pt}
\makeatletter
\rhead{\footnotesize \@title}
\cfoot{\footnotesize Strona \thepage \hspace{1pt} z \pageref{LastPage}}
\makeatother

% dotted lines in ToC
\usepackage{tocloft} % for dots in ToC
\renewcommand{\cftsecleader}{\cftdotfill{\cftdotsep}}
\renewcommand{\cftsecaftersnum}{.}
\renewcommand{\cftsecnumwidth}{1.75em}

% dots in bookmarks
\usepackage{bookmark}
\bookmarksetup{
	numbered,
	open
}

% prefer no hypenation
\righthyphenmin=62
\lefthyphenmin=62

% don't hypenate sections
\usepackage[
	%raggedright
]{titlesec}
\titlelabel{\thetitle.\enskip}
% smaller font for title dots in section titles
% \titleformat{\section}{\normalfont\fontsize{12}{15}\bfseries}{\thesection.}{0.75em}{}
% \titleformat{\subsection}{\normalfont\fontsize{9}{12}\bfseries}{\thesubsection.}{0.75em}{}

% define styles
\theoremstyle{definition}
\newtheorem{definition}{Definicja}[section]
\newtheoremstyle{case}{}{}{}{}{}{:}{ }{}
\theoremstyle{case}
\newtheorem{case}{Przypadek}

% helper
% \newcommand*\from{ \colon }
\let\iff\Leftrightarrow
\let\temp\phi
\let\phi\varphi
\let\varphi\temp

% define title
\title{Zagadnienia z analizy}

\renewcommand{\baselinestretch}{1.2}

\begin{document}

% bookmark title page
\makeatletter
\pdfbookmark[0]{\@title}{title}
\makeatother

% title page
\pagenumbering{gobble}
\begin{titlepage}
	\makeatletter
	% \vfill
	\begin{center}
		\vspace*{0.3\textheight}
		{\LARGE
			\@title
		\par}
	\end{center}\par
	\vfill
	\tableofcontents
	\vfill
	\makeatother
\end{titlepage}

\newpage

\pagenumbering{arabic}
\setcounter{page}{1}

\setcounter{section}{63}

\section{Podać i udowodnić zasadnicze twierdzenie rachunku całkowego,
które pozwala obliczyć całkę Reimanna za pomocą funkcji pierwotnej.}
\indent

Jeżeli ${f \colon [a,b] \to \mathbb{R}}$ jest funkcją ciągłą, natomiast ${F \colon [a,b] \to \mathbb{R}}$
jest jej dowolną funkcją pierwotną, to zachodzi równość:
\begin{equation*}
	\int _a^b f(x) \mathrm d x = [ F(x) ] _a ^b = F(b) - F(a)
\end{equation*}

\begin{proof}
	Zdefiniujmy funkcję $F$ dla każdego ${x \in [a,b]}$:
	\begin{equation*}
		F(x) = \textstyle \int_a^x f(t) \mathrm d t
	\end{equation*}

	Skoro funkcja $f$ jest ciągła, to na mocy twierdzenia o funkcji
	górnej granicy całkowania funkcja $F$ jest różniczkowalna
	i zachodzi równość ${F^\prime(x)=f(x)}$ we wszystkich punktach ${x \in (a,b)}$.
	Oznacza to, że $F$ jest funkcją pierwotną funkcji $f$.
	Ponieważ każde dwie funkcje pierwotne danej funkcji różnią się o stałą,
	to dla pewnej liczby rzeczywistej C oraz dowolnego ${x \in [a,b]}$
	zachodzi równość
	\begin{equation}
		\label{eq.1}
		g(x)=F(x)+C
	\end{equation}
	Z definicji funkcji $F$ wynika, że
	\begin{equation}
		\label{eq.2}
		\textstyle \int_a^b f(t) \mathrm d t = F(b)
	\end{equation}
	A skoro
	\begin{equation}
		\label{eq.3}
		F(a) = \textstyle \int_a^a f(t) \mathrm d t=0
	\end{equation}
	to możemy kontynuować obliczenia, zapisując
	${F(b)=F(b)-F(a)=(F(b)+C)-(F(a)+C)=g(b)-g(a)}$,
	gdzie ostatnia równość wynika z \eqref{eq.1}. Połączenie \eqref{eq.2} z \eqref{eq.3} implikuje żądany wzór i kończy dowód twierdzenia.
\end{proof}

\section{Co to jest całka niewłaściwa w sensie Reimanna?}

\subsection{Całka niewłaściwa Riemanna w przedziale $[a,b)$ lub $(a,b]$}
Niech ${f \colon [a,b) \to \mathbb{R}}$ będzie funkcją całkowalną w sensie Riemmana
na każdym z przedziałów domkniętych ${[a,\beta]}$,
gdzie $a < \beta < b$.
Załóżmy, że funkcja $f$jest nieograniczona w pewnym lewostronnym sąsiedztwie punktu $b$.
\begin{definition}
	\label{def.1}
	Całką niewłaściwą Riemanna funkcji $f$ nazywamy granicę:
	\begin{equation*}
		\lim_{\beta \to b^-} \int _a ^{\beta} f(x) \mathrm d x
		\textnormal{\enskip i oznaczamy ją symbolem \enskip}
		\int_a^b f(x) \mathrm d x
		\textnormal{.}
	\end{equation*}
\end{definition}

Jeżeli granica istnieje i jest skończona, to mówimy,
że całka niewłaściwa ${\int_a^b f(x) \mathrm d x }$ jest zbieżna.
Natomiast jeżeli granica ta nie istnieje lub jest niewłaściwa, to mówimy,
że całka niewłaściwa ${\int_a^b f(x) \mathrm d x}$ jest rozbieżna.

W analogiczny sposób definuje się całkę niewłaściwą Riemanna
${\int_a^b f(x) \mathrm d x}$ funkcji $f$ określonej na przedziale ${(a,b]}$,
jak również pojęcia jej zbieżności i rozbieżności.
Przyjmujemy wówczas, że
\begin{equation*}
\int_a^b f(x) \mathrm d x \coloneqq \lim_{\alpha \to a^+} \int_{\alpha}^b f(x) \mathrm d x
\end{equation*}

\subsection{Całka niewłaściwa Reimanna w przedziale $(a,b)$}
Niech ${f \colon (a, b) \to \mathbb{R}}$ będzie funkcją całkowalną w sensie Riemanna
w każdym przedziale domkniętym $[\alpha,\beta]$, przy czym $a < \alpha < \beta < b$.
Załóżmy, że funkcja $f$ jest nieograniczona w pewnym prawostronnym sąsiedztwie punktu $a$ oraz
w pewnym lewostronnym sąsiedztwie punktu $b$.
\begin{definition}
	\label{def.2}
	Całkę niewłaściwą Riemanna funkcji $f$ w $(a,b)$ definujemy jako
	\begin{equation*}
		\int_a^b f(x)dx
		\coloneqq
		\int_a^c f(x) \mathrm d x + \int_c^b f(x) \mathrm d x
		\textnormal{,\enskip gdzie $c$ jest dowolnie wybranym punktem z $(a,b)$.}
	\end{equation*}
\end{definition}
Jeżeli obie całki w powyższej sumie są zbieżne, to mówimy,
że całka $\int_a^b f(x) \mathrm d x$ jest zbieżna.
Gdy któraś z tych całek nie istnieje lub jest rozbieżna, to mówimy,
że całka $\int_a^b f(x) \mathrm d x$ jest rozbieżna.

Jeżeli całka $\int_a^b f(x) \mathrm d x$ jest zbieżna, to jej wartość nie zależy od wyboru punktu $c$.

\subsection{Całka niewłaściwa Reimanna w przedziale
	\texorpdfstring{$(-\infty,b]$}{(-inf,b]},
	\texorpdfstring{$[a,\infty)$}{[a,+inf)} lub
	\texorpdfstring{$(-\infty,\infty)$}{(-inf,+inf)}
}
W powyższych definicjach \ref{def.1} i \ref{def.2} można podstawić $-\infty$ w miejsce $a$ i/lub $\infty$ w miejsce $b$:
\begin{itemize}
	\item
		$\int\limits_a^{\infty} f(x) \mathrm d x
		\coloneqq
		\lim\limits_{\beta \to \infty} \int\limits_a^{\beta} f(x) \mathrm d x$
	\item
		$\int\limits_{-\infty}^b f(x) \mathrm d x
		\coloneqq
		\lim\limits_{\alpha \to -\infty} \int\limits_{\alpha}^b f(x) \mathrm d x$
	\item
		$\int\limits_{-\infty}^{\infty} f(x) \mathrm d x
		\coloneqq
		\int\limits_{-\infty}^{c} f(x) \mathrm d x + \int\limits_{c}^{\infty} f(x) \mathrm d x$ ($c \in \mathbb{R}$).
\end{itemize}

\section{Obliczyć całkę niewłaściwą funkcji
	\texorpdfstring{$ 1 / x^p $}{1/x\textasciicircum p}
	w przedziale
	\texorpdfstring{$ [ 1, \infty ) $}{[1,inf)}.
	Dla jakich wartości parametru $p$ całka ta jest zbieżna?}

% \subsection{Przypadek $p=1$}
Przypadek $p=1$:
\begin{equation*}
	\int_1^\infty \frac{\mathrm d x}{x}
	= \lim_{\beta \to \infty} \left. \ln x \right|_1^{\beta}
	= \lim_{\beta \to \infty} (\ln \beta -\ln 1)
	= \infty
\end{equation*}

% \subsection{Przypadek \texorpdfstring{$p\neq1$}{p!=1}}
Przypadek $p\neq 1$:
\begin{equation*}
	\int_1^{\infty}\frac{\mathrm d x}{x^p}
	= \int_1^{\infty} x^{-p} \mathrm d x
	= \lim_{\beta \to \infty}\int_1^\beta x^{-p} \mathrm d x
	= \lim_{\beta \to \infty} \left. \frac{x^{-p+1}}{-p+1} \right|_1^\beta
	= \lim_{\beta \to \inf} \left. \frac{1}{(1-p)x^{p-1}} \right|_1^\beta
	= \frac{1}{1-p} \lim_{\beta \to \inf} \left( \frac{1}{\beta^{p-1}} - \frac{1}{1^{p-1}} \right)
\end{equation*}

Zauważmy, że
\begin{equation*}
	\lim_{\beta \to \infty} \frac{1}{\beta^{p-1}}
	= \left\{
		\begin{array}{ll}
			\infty & \text{ gdy } p-1< 0
			\\ 0 & \text{ gdy } p-1> 0
		\end{array}
	\right.
\end{equation*}

A zatem
\begin{equation*}
	\int_1^\infty \frac{\mathrm d x}{x^p}
	= \left\{
		\begin{array}{ll}
			\infty & \text{ gdy } p< 1
			\\ \frac{1}{(p-1)a^{p-1}} & \text{ gdy } p> 1
		\end{array}
	\right.
\end{equation*}

\textbf{Reasumując}
\begin{equation*}
\int_1^\infty \frac{\mathrm d x}{x^p}
= \left\{
	% \begin{array}{ll}
	% 	\textnormal{ jest zbieżna dla } p>1
	% 	\\ \textnormal{ jest rozbieżna dla } p\leq1
	% \end{array}
	\begin{array}{lll}
		\infty, & \textnormal{ (jest rozbieżna) } & \text{ gdy } p\leq1
		\\ \frac{1}{(p-1)a^{p-1}}, & \textnormal{ (jest zbieżna) } & \text{ gdy } p>1
	\end{array}
\right.
\end{equation*}

\section{Podać i udowodnić kryterium całkowe zbieżności szeregu.
	Zastosować to kryterium do szeregu harmonicznego z wykładnikiem
	\texorpdfstring{$ \alpha $}{a}.}
	\subsection{Teoria}
	Niech $f \colon [n_0,\infty) \to \mathbb{R}$, $f(n)$ jest nierosnąca i nieujemna dla $n> n_0$.
		[Analogicznie dla niemalejącej i niedodatniej.]
	\begin{equation*}
		\textnormal{Wówczas \enskip}
		\sum_{n={n_0}}^\infty f(n) \textnormal{ \enskip jest zbieżny}
		\iff
		\int_{n_0}^\infty f(n) \mathrm d n \textnormal{ \enskip jest zbieżna}
	\end{equation*}
	\begin{proof}%{\textit{(z wykładu)}}
		Weźmy $a_n \coloneqq \int_n^{n+1} f(x) \mathrm d x$ ($n \geq n_0$, $n \in \mathbb{N}$).
		\\Mamy: $a_n \leq f(n) \leq a_{n-1}$.
		\\Stąd:
		\begin{equation*}
		\begin{CD}
			\displaystyle\int_{n_0}^{n+1}f(x) \mathrm d x @.
			\quad \leq \quad @.
			\displaystyle\sum_{i=n_0}^nf(i) @.
			\quad \leq \quad @.
			f(n_0) + @. \underbrace{\int_{n_0}^nf(x) \mathrm d x}
		\\ @VV{n \to \infty}V @. @VV{n \to \infty}V @. @. @VV{n \to \infty}V
		\\ \displaystyle\int_{n_0}^\infty f(x) \mathrm d x @.
		\quad \leq \quad @.
		\displaystyle\sum_{i=n_0}^\infty f(i) @.
		\leq @.
		@. \displaystyle\int_{n_0}^\infty f(x) \mathrm d x
		\end{CD}
		\end{equation*}
	\end{proof}

	\subsection{Przykład}

	\begin{gather*}
		\alpha > 0 ,\quad \displaystyle\sum_{n=1}^\infty \frac{1}{n^\alpha} ,\quad f(x) = \frac{1}{x^\alpha}
		\\\int_1^\infty \frac{\mathrm d x}{x^\alpha}
		= \left. \frac{x^{1-\alpha}}{1-\alpha} \right|_1^\infty
		= \left\{
			\begin{array}{lr}
				\frac{1}{1-\alpha} & \alpha > 1
				\\\frac{\infty-1}{1-\alpha} & \alpha < 1
			\end{array}
			\right.
	\end{gather*}
	Szereg jest zbieżny $\iff \alpha > 1$


\section{Podać wzór na długość krzywej zadanej parametrycznie.}
Niech $-\infty < a < b < \infty$ oraz
$f_1, f_2, \dots, f_n \colon [a,b] \to \mathbb{R}$ będą funkcjami ciągłymi klasy $C^1$.
Krzywą $\gamma$ nazywamy zbiór punktów
${\gamma = \{ (x_1, x_2, \dots, x_n) \in \mathbb{R}^n \colon 
x_1 = f_1 (t), x_2 = f_2 (t), \dots, x_n = f_n (t), t \in [a,b] \} }$.

Wówczas długość krzywej wyraża się wzorem:
\begin{equation*}
	| \gamma | = \int_a^b \sqrt{x_1^\prime(t)^2 + x_2^\prime(t)^2 + \dots + x_n^\prime(t)^2} \mathrm d t
\end{equation*}

W szczególności, dla $n=2$, $x_1=t$, $x_2=f(t)$:
\begin{equation*}
	| \gamma | = \int_a^b \sqrt{1 + f ^\prime (t) ^2} \mathrm d t
\end{equation*}

\section{Podać wzór na objętość i pole powierzchni bryły obrotowej.}
% \pagedepth\maxdimen
\begin{wrapfigure}{R}{20em+2em}
	\def\svgwidth{20em}
	\input{obrot_bryly.pdf_tex}
\end{wrapfigure}
$V$ - bryła powstała przez obrót wykresu $y=f(x)$ wokół osi Ox. $f \colon [a,b] \to [0,\infty)$.
\begin{equation*}
	\textnormal{objętość: \enskip}
	| V | = \pi \int_a^b f(x)^2 \mathrm d x
\end{equation*}
\begin{equation*}
	\textnormal{pole} (V) = 2\pi \int_a^b \sqrt{1 + f^\prime(x)^2} \mathrm d x
\end{equation*}

\end{document}
