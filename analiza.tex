\documentclass{article}
% packages
% page geometry
\usepackage[
	a4paper,
	margin=1.5cm,
	centering,
	nohead,
	includefoot
]{geometry}
% cols
\usepackage{multicol}
% utf8 support
\usepackage[utf8]{inputenc}
% because...
\usepackage[T1]{fontenc}
% language support
\usepackage[polish]{babel}
% indent fist paragraph
\usepackage{indentfirst}
% better math
\usepackage{amsmath}
% math stuff
\usepackage{mathtools}
\usepackage{amscd}
% proofs
\usepackage{amsthm}
\renewcommand{\qedsymbol}{\textsquare}
\numberwithin{equation}{section}

%different font
\usepackage[p]{newtxtext}
\usepackage[slantedGreek,smallerops,upint,cmbraces]{newtxmath}
\renewcommand{\familydefault}{\sfdefault}

% for images
\usepackage{graphicx}
% for placement
\usepackage{wrapfig}
% plots
\usepackage{tikz}
\usetikzlibrary{patterns}

% hyperlinks formatting
\usepackage[
	colorlinks,
	linkcolor=black,
	urlcolor=black,
	citecolor=black,
	plainpages=false,
	pdfpagelabels,
	linktoc=all,
	unicode,
	backref
]{hyperref}
\makeatletter
\renewcommand{\Hy@numberline}[1]{#1. }
\makeatother

% for \LastPage
\usepackage{lastpage}

%custom headers and footers
\usepackage{fancyhdr}
\pagestyle{fancy}
\fancyhf{} % clear all header and footer fields
\renewcommand{\headrulewidth}{0pt}
\cfoot{\footnotesize Strona \thepage \hspace{1pt} z \pageref{LastPage}}

% dotted lines in ToC
\usepackage[titles]{tocloft} % for dots in ToC
\renewcommand{\cftsecleader}{\cftdotfill{\cftdotsep}} % dotted line
\renewcommand{\cftsecaftersnum}{.} % Dots adter numbers
\renewcommand{\cftsecnumwidth}{1.75em} % wider space for numbers

% dots in bookmarks
\usepackage{bookmark}
\bookmarksetup{
	numbered,
	open
}

% better lists
\usepackage{enumitem}
\setlist{
	noitemsep,
	topsep=\parskip
}

% prefer no hypenation
\righthyphenmin=62
\lefthyphenmin=62
\hyphenpenalty=10000
\exhyphenpenalty=10000
\emergencystretch 4em

% don't hypenate sections
\usepackage[nobottomtitles*]{titlesec}
\titlelabel{\thetitle.\enskip}
% smaller font for title dots in section titles
% \titleformat{\section}{\normalfont\fontsize{12}{15}\bfseries}{\thesection.}{0.75em}{}
% \titleformat{\subsection}{\normalfont\fontsize{9}{12}\bfseries}{\thesubsection.}{0.75em}{}

% define styles
\theoremstyle{definition}
\newtheorem{definition}{Definicja}[section]
\newtheoremstyle{case}{}{}{}{}{}{:}{ }{}
\theoremstyle{case}
\newtheorem{case}{Przypadek}

% helper
% \newcommand*\from{ \colon }

\let\implies\Rightarrow
\let\iff\Leftrightarrow

\let\temp\phi
\let\phi\varphi
\let\varphi\temp

\let\oldforall\forall
\let\forall\undefined
\DeclareMathOperator{\forall}{\mkern2mu\oldforall}

\let\oldexists\exists
\let\exists\undefined
\DeclareMathOperator{\exists}{\mkern2mu\oldexists}

\DeclarePairedDelimiter\abs{\lvert}{\rvert}%
\DeclarePairedDelimiter\norm{\lVert}{\rVert}%
% Swap the definition of \abs* and \norm*, so that \abs
% and \norm resizes the size of the brackets, and the starred version does not.
\makeatletter
\let\oldabs\abs
\def\abs{\@ifstar{\oldabs}{\oldabs*}}
\let\oldnorm\norm
\def\norm{\@ifstar{\oldnorm}{\oldnorm*}}
\makeatother

\makeatletter
\g@addto@macro\@floatboxreset\centering
\makeatother

\DeclareMathOperator{\tg}{tg}
\DeclareMathOperator{\ctg}{ctg}

% define title
\title{Zagadnienia z analizy}

\renewcommand{\baselinestretch}{1.2}
\renewcommand\arraystretch{1.2}

\begin{document}
\thispagestyle{empty}

% bookmark title page
\makeatletter
\pdfbookmark[0]{\@title}{title}
\makeatother

% title page
\pagenumbering{Alph}
\begin{titlepage}
	\vspace*{\fill}
	\begin{center}
		\makeatletter
		\huge
		\@title
		\makeatother
	\end{center}
	\vspace*{\fill}
	\tableofcontents{
		\thispagestyle{empty}
	}
\end{titlepage}

\newpage

\pagenumbering{arabic}
\setcounter{page}{1}

\section{Co to jest funkcja? dziedzina funkcji? zbiór wartości funkcji?}
Niech $X$, $Y$ będą zbiorami niepustymi.
\subsection{Funkcja}
Funkcją przekształcającą zbiór $X$ w zbiór $Y$ nazywamy dowolny podzbiór $f \subseteq X \times Y$ taki, że dla każdego $x \in X$ istnieje dokładnie jeden $y \in Y$, dla którego $(x,y)\in f$.
Oznaczamy to jako $f \colon X \to Y$. Funkcję nazywamy również przekształceniem lub przyporządkowaniem.

Relację $f \subseteq X \times Y$ nazywamy funkcją jeżeli:
\begin{itemize}
	\item $\forall x \in X \exists y \in Y \colon (x,y) \in f$
	\item $\forall x \in X \forall y \in Y \forall z \in Z \colon (x,y)\in f \land (x,z)\in f \implies y = z$
\end{itemize}

\subsection{Dziedzina}
Zbiór $D_f \subseteq X$ nazywamy dziedziną funkcji $f$, jeżeli spełnia warunek:
$x \in D_f \iff \exists y \in Y \colon y = f(x)$

\subsection{Przeciwdziedzina}
Zbiór $f(x) \subseteq Y$ nazywamy zbiorem wartości funkcji $f$, jeżeli spełnia warunek:
$y \in f(X) \iff \exists x \in X \colon y = f(x)$

\section{Co to jest złożenie funkcji? funkcja odwrotna?}
Jeżeli $f \colon X \to Y$ oraz $g \colon Y \to Z$ są funkcjami, to ich złożeniem nazywamy funkcję $h \colon X \to Z$ taką, że ${h(x) = g(f(x))}$ dla $x \in X$. Złożenie dwóch funkcji oznaczamy $h=g \circ f$, dla dowolnego $x$ z dziedziny funkcji $f$ mamy równość $h(x)=g(f(x))=(g \circ f)(x)$.

Jeżeli $f \colon X \to Y$ jest bijekcją, to można zdefiniować funkcję odwrotną $f^{-1} \colon Y \to X$:
${f^{-1}(y)=x \iff f(x)=y}$.
Przykład: $y=a^x$ - odwrotna: $y = \log_a{x}$

\section{Co nazywamy obrazem, przeciwobrazem zbioru przez funkcję? Podać ich własności.}
Niech $f \colon X \to Y$.
\subsection{Obraz}
Obrazem zbioru $A \subseteq X$ nazywamy zbiór ${f(A) = \{y \in Y \colon \exists x \in A \colon y=f(x)\}}$.
\\Własności ($A, B \subseteq X$):
\begin{itemize}
	\item monotoniczność: $A \subseteq B \implies f(A) \subseteq f(B)$
	\item $f(A \cup B) = f(A) \cup f(B)$
	\item $f(A \cap B) \subseteq f(A) \cap f(B)$ (równość jeżeli $f$ jest różnowartościowa)
	\item $f(A)\setminus f(B) \subseteq f(A \setminus B)$ (równość jeżeli $f$ jest różnowartościowa)
\end{itemize}

\subsection{Przecwiobraz}
Przeciwobrazem zbioru $C \subseteq Y$ nazywamy zbiór: ${f^{-1}(C) = \{ x \in X \colon f(x) \in C \}}$.
\\Własności ($C, D \subseteq Y$):
\begin{itemize}
	\item monotoniczność: $C \subseteq D \implies f^{-1}(C) \subseteq f^{-1}(D)$
	\item $f^{-1}(C \cup D) = f^{-1}(C) \cup f^{-1}(D)$
	\item $f^{-1}(C \cap D) = f^{-1}(C) \cap f^{-1}(D)$
	\item $f^{-1}(C) \setminus f^{-1}(D) = f^{-1}(C \setminus D)$
	\item $f^{-1}(Y \setminus D) = X \setminus f^{-1}(D)$
\end{itemize}

\section{Co to jest iniekcja, suriekcja, bijekcja?}
Iniekcja: ${\forall x,y \in X \colon x \neq y \implies f(x) \neq f(y)}$ (Równoważnie: $f(x) = f(y) \iff x = y$)

Suriekcja: ${\forall y \in Y \exists x \in X \colon f(x)=y}$

Bijekcja: funkcja jest iniekcją i suriekcją.

\section{Naszkicować wykresy funkcji wykładniczych i logarytmicznych.}
\begin{multicols}{2}
	\centering

	\begin{tikzpicture}
			[
				every circle/.style={radius=.05},
				declare function={
					r={3};
				}
			]
		\draw [domain=-r:r] plot (\x, {r^(1/r)^\x});
		\draw [fill] (0,1) node [right] {$1$} circle;
		\draw [thick,->]
			(-r,0) -- (r,0) node [below] {$x$};
		\draw [thick,->]
			(0,0) -- (0,r) node [right] {$y$};
	\end{tikzpicture}
	\\$y=a^x$, $a>1$

	\begin{tikzpicture}
			[
				every circle/.style={radius=.05},
				declare function={
					r={3};
				}
			]
		\draw [domain=-r:r] plot (\x, {r^(-1/r)^\x});
		\draw [fill] (0,1) node [right] {$1$} circle;
		\draw [thick,->]
			(-r,0) -- (r,0) node [below] {$x$};
		\draw [thick,->]
			(0,0) -- (0,r) node [right] {$y$};
	\end{tikzpicture}
	\\$y=a^x$, $a \in (0,1)$
\end{multicols}

\begin{multicols}{2}
	\centering

	\begin{tikzpicture}
			[
				every circle/.style={radius=.05},
				declare function={
					r={3};
				}
			]
		\draw [domain={1/r}:r] plot (\x, {ln(\x)/ln(r^(1/r))});
		\draw [fill] (1,0) node [below] {$1$} circle;
		\draw [thick,->]
			(0,0) -- (r,0) node [below] {$x$};
		\draw [thick,->]
			(0,-r) -- (0,r) node [right] {$y$};
	\end{tikzpicture}
	\\$y=\log_a(x)$, $a>1$

	\begin{tikzpicture}
			[
				every circle/.style={radius=.05},
				declare function={
					r={3};
				}
			]
		\draw [domain={1/r}:r] plot (\x, {ln(\x)/ln(r^(-1/r))});
		\draw [fill] (1,0) node [below] {$1$} circle;
		\draw [thick,->]
			(0,0) -- (r,0) node [below] {$x$};
		\draw [thick,->]
			(0,-r) -- (0,r) node [right] {$y$};
	\end{tikzpicture}
	\\$y=\log_a(x)$, $a \in (0,1)$
\end{multicols}

\section{Naszkicować wykresy funkcji trygonometrycznych i cyklometrycznych.}
\begin{multicols}{2}
	\centering
	\begin{tikzpicture}
			[
				every circle/.style={radius=.05},
				declare function={
					r={2.1};
				}
			]
		\draw [
			%blue,
				domain={-0.5*pi}:{r*pi},
				samples=100
			]
			plot (\x, {sin(deg(\x))})
			node [right] {$y=sin(x)$};
		\draw
			[
			%green,
				domain={-0.5*pi}:{r*pi},
				samples=100
			]
			plot (\x, {cos(deg(\x))})
			node [right] {$y=cos(x)$};

		\draw [fill] (0,1) node [right] {$1$} circle;
		\draw [fill] (0,-1) node [right] {$-1$} circle;

		\draw [fill] (pi,0) node [below] {$\pi$} circle;
		\draw [fill] (2*pi,0) node [below] {$2\pi$} circle;

		\draw [fill] (pi/2,0) node [below] {$\frac{\pi}{2}$} circle;
		\draw [fill] (3*pi/2,0) node [below] {$\frac{3\pi}{2}$} circle;

		\draw [fill] (pi/4,0) node [below] {$\frac{\pi}{4}$} circle;
		\draw [fill] (3*pi/4,0) node [below] {$\frac{3\pi}{4}$} circle;
		\draw [fill] (5*pi/4,0) node [below] {$\frac{5\pi}{4}$} circle;
		\draw [fill] (7*pi/4,0) node [below] {$\frac{7\pi}{4}$} circle;

		\draw [thick,->]
			(-0.5*pi,0) -- (r*pi,0) node [below] {$x$};
		\draw [thick,->]
			(0,-1.5) -- (0,1.5) node [right] {$y$};
	\end{tikzpicture}

	\begin{tikzpicture}
			[
				every circle/.style={radius=.05},
			]

		\draw [
			%color=blue,
				domain=-1:1,
				samples=100
			]
			plot (\x, {rad(asin(\x))})
			node [right] {$y=\arcsin(x)$};
		\draw [
			%color=green,
				domain=-1:1,
				samples=100
			]
			plot (\x, {rad(acos(\x))})
			node [above right] {$y=\arccos(x)$};

		\draw [fill] (1,0) node [below] {$1$} circle;
		\draw [fill] (-1,0) node [below] {$-1$} circle;

		\draw [fill] (0,-pi/2) node [right] {$-\frac{\pi}{2}$} circle;
		\draw [fill] (0,pi/2) node [right] {$\frac{\pi}{2}$} circle;
		\draw [fill] (0,pi) node [right] {$\pi$} circle;

		\draw [thick,->]
			(-1.5,0) -- (1.5,0) node [below] {x};
		\draw [thick,->]
			(0,-2) -- (0,4) node [right] {y};
	\end{tikzpicture}

	\begin{tikzpicture}
			[
				every circle/.style={radius=.05},
				declare function={
					limit=0.1;
				}
			]

		\draw [
		%color=blue,
				domain={-pi/2+limit}:{pi/2-limit},
				samples=100
			]
			plot (\x, {tan(deg(\x))/3})
			node [right] {$y=\tg(x)$};
		\draw [
			%color=green,
				domain={0+limit}:{pi-limit},
				samples=100
			]
			plot (\x, {cot(deg(\x))/3})
			node [right] {$y=\ctg(x)$};

		\draw [fill] (-pi/2,0) node [below] {$-\frac{\pi}{2}$} circle;
		\draw [fill] (pi/2,0) node [below] {$\frac{\pi}{2}$} circle;
		\draw [fill] (pi,0) node [below] {$\pi$} circle;
		\draw [dotted] (-pi/2,-4) -- (-pi/2,4);
		\draw [dotted] (pi/2,-4) -- (pi/2,4);
		\draw [dotted] (pi,-4) -- (pi,4);

		\draw [thick,->]
			(-2,0) -- (3.5,0) node [below] {x};
		\draw [thick,->]
			(0,-4) -- (0,4) node [right] {y};
	\end{tikzpicture}
	%TODO: arctg arcctg
\end{multicols}

\section{Co to jest homografia? Funkcja wymierna?}
Funkcję $f(x) = \frac{ax+b}{cx+d}$, gdzie $a,b,c,d \in \mathbb{C}$ i $ad-bc \neq 0$, nazywamy funkcją homograficzną. Szczególnym przypadkiem homografii jest funkcja w postaci $f(x) = \frac{a}{x}$.

Jeżeli funkcje $g(x)$ i $h(x) \neq 0$ są wielomianami, to funkcję $f(x)=\frac{g(x)}{h(x)}$ nazywa się funkcją wymierną.

\section{Wykazać, że odwzorowaniem odwrotnym do homografii jest homografia.}
$f \colon \mathbb{C} \to \mathbb{C}$, $a,b,c,d \in \mathbb{C}$, $ad-bc \neq 0$, $f(z) = \frac{az+b}{cz+d}$
\begin{align*}
	x &= \frac{az+b}{cz+d}
	\\ xcz+xd &= b-xd
	\\ zxc-az &= b-xd
	\\ z(xc-a) &= b-xd
	\\ z &= \frac{b-xd}{xc-a}
	\\ z &= \frac{-xd+b}{xc-a}
	\\ &\Downarrow
	\\ f^{-1}(z) &= \frac{-dz+b}{cz-a}
\end{align*}

\section{Podać zasadę indukcji matematycznej.}
Jeżeli:
\begin{enumerate}
	\item Istnieje taka liczba naturalna $n_0$, że $T(n_0)$ jest zdaniem prawdziwym
	\item Dla każdej liczby naturalnej $n>n_0$ prawdziwa jest implikacja $T(n) \implies T(n+1)$
\end{enumerate}
to $T(n)$ jest zdaniem prawdziwym dla każdej liczby naturalnej $n \geq n_0$.

\section{Podać i udowodnić (indukcyjnie) nierówność Bernoulliego.}
\begin{multicols}{2}
	Jeśli $x \geq -1$:
	\begin{equation*}
		\begin{cases}
			(1+x)^\alpha \leq 1 + \alpha x & \textnormal{dla\:} 0 < \alpha \leq 1
			\\ (1+x)^\alpha \geq 1 + \alpha x & \textnormal{dla\:} \alpha \geq 1
		\end{cases}
	\end{equation*}
	\vspace*{\fill}

	Równość zachodzi dla $\alpha = 1$.
\end{multicols}
\begin{proof}
	\begin{enumerate}
		\item
			Dla $\alpha = 1$:
			\begin{equation*}
				(1+x)^1 = 1+x = 1 + 1x
			\end{equation*}
		\item
			Istnieje $k \geq 1$ takie, że
			\begin{equation}
				\label{eq.ber}
				(1+x)^k \geq 1 + kx
			\end{equation}
		\item
			% \begin{equation*}
			% 	(1+x)^{k+1} \geq 1+(k+1)x
			% \end{equation*}
			\begin{align*}
				(1+x)^{k+1}
				   &= (1+x)^k (1+x)
				\\ &\geq (1+kx)(1+x) &\textnormal{\quad z założenia (wzór \ref{eq.ber})}
				\\ &= 1+x+kx+kx^2
				\\ &= 1+(1+k)x+kx^2
				\\ &\geq 1+(1+k)x &\textnormal{\quad $k\geq 1$, więc $kx^2\geq 0$}
			\end{align*}
	\end{enumerate}
\end{proof}

\section{Podać i udowodnić wzór dwumianowy Newtona.}
\begin{multicols}{2}
	\begin{equation*}
		(x+y)^n = \sum_{k=0}^n \binom{n}{k} x^{n-k}y^k
	\end{equation*}
	\vspace*{\fill}

	Przykład: $(x+y)^2=x^2+2xy+y^2$
\end{multicols}
\begin{proof}
	\begin{enumerate}
		\item
			$n=1$:
			\begin{equation*}
				(x+y)^1=x+y=\binom{1}{0}xy^0+\binom{1}{1}x^0y=\sum_{k=0}^1\binom{1}{k}x^{1-k}y^k
			\end{equation*}
		\item
			\begin{equation}
				\label{eq.dwumian}
				(x+y)^n=\sum_{k=0}^n\binom{n}{k}x^{n-k}y^k
			\end{equation}
		\item
			% \begin{equation*}
			% 	(x+y)^{n+1}=\sum_{k=0}^{n+1}\binom{n+1}{k}x^{n+1-k}y^k
			% \end{equation*}
			\begin{align*}
				(x+y)^{n+1} &=
				(x+y)(x+y)^n
				\\ &= (x+y)\sum_{k=0}^n\binom{n}{k}x^{n-k}y^k &\textnormal{\quad z założenia (wzór \ref{eq.dwumian})}
				\\ &= \sum_{k=0}^n\binom{n}{k}x^{n-k+1}y^k + \sum_{k=0}^n\binom{n}{k}x^{n-k}y^{k+1}
				\\ &= \binom{n}{0}x^{n+1}y^0
				+ \sum_{k=1}^n \left(\binom{n}{k}+\binom{n}{k-n}\right)x^{n-k+1}y^k
				+ \binom{n}{n}x^0y^{n+1}
				\\ &= \binom{n+1}{0}x^{n+1}
				+ \sum_{k=1}^n\binom{n+1}{k}x^{(n+1)-k}y^k
				+ \binom{n+1}{n+1}y^{n+1}
				\\ &= \sum_{k=0}^{n+1}\binom{n+1}{k}x^{(n+1)-k}y^k
			\end{align*}
	\end{enumerate}
\end{proof}

\section{Co to jest trójkąt Pascala i do czego służy? Wykazać, że
	\texorpdfstring
	{${\binom{n}{t}+\binom{n}{t-1}=\binom{n+1}{t}}$.}
		{(n t)+(n t-1)=(n+1 t).}
	}
Trójkąt Pascala - trójkątna tablica liczb.
\begin{multicols}{2}
	TODO: narysuj
	\\
	Liczby znajdujące się w n-tym wierszu to kolejne współczynniki dwumianu Newtona. Służy do obrazowania liczby dróg którymi można dość od wierzchołka do danego punku (rachunek prawdopodobieństwa).
\end{multicols}
\begin{proof}
	\begin{align*}
		\binom{n}{t}+\binom{n}{t-1}
		&= \frac{n!}{t!(n-t)!}+\frac{n!}{(t-1)!(n-t+1)!}
		\\ &= \frac{n!}{t(t-1)!(n-t)!}+\frac{n!}{(t-1)!(n-t)!(n-t+1)}
		\\ &= \frac{n!(n-t+1)+n!t}{t(t-1)!(n-1)!(n-t+1)}
		\\ &= \frac{n!(n-t+1+t)}{t!(n-t+1)!}
		\\ &= \frac{(n+1)!}{t!((n+1)-t)!}
		\\ &= \binom{n+1}{t}
	\end{align*}
\end{proof}

\section{Podać aksjomat ciągłości dla zbioru liczb rzeczywistych. Co to jest kres dolny, górny podzbioru liczb rzeczywistych?}
Aksjomat ciągłości postuluje, że każdy niepusty i ograniczony z góry podzbiór zbioru liczb rzeczywistych ma kres górny.

Kres górny/dolny - najmniejsze z ograniczeń górnych/największe z ograniczeń dolnych danego zbioru, o ile ograniczenia istnieją.
\\Niech $A \subseteq \mathbb{R}$. Liczbę $\beta \in \mathbb{R}$ nazywamy kresem górnym zbioru $A$, jeżeli $\beta$ jest największą liczbą zbioru $A$ lub $\beta$ jest najmniejszą spośród liczb ograniczających z góry. Analogicznie kres dolny.

\section{Podać definicję ciągu. Co to jest ciąg zbieżny, rozbieżny do \texorpdfstring{$+\infty$,$-\infty$}{+inf,-inf}?}
Ciągiem nazywamy funkcję, której dziedziną jest zbiór liczb naturalnych $\mathbb{N}$ lub jego skończony odcinek początkowy ($\{1,2,3,\dots,n\}$). W 1. przypadku jest to ciąg nieskończony, w 2. - ciąg skończony n-elementowy.

Ciąg zbieżny - ciąg, który posiada granicę: $\lim\limits_{n\to \infty}a_n=g\in \mathbb{R}$.
$\forall \epsilon > 0 \exists n \in \mathbb{N} \colon \abs{a_n-g}<\epsilon$ (tzn. $a_n \in (g-\epsilon,g+\epsilon)$).

Ciąg rozbieżny:
$\lim\limits_{n \to \infty}a_n = {+ \atop -} \infty \iff \forall M \in \mathbb{R} \exists N \in \mathbb{N} \forall n > N \colon a_n {> \atop <} M$.

\setcounter{section}{63}
\section{Podać i udowodnić zasadnicze twierdzenie rachunku całkowego,
które pozwala obliczyć całkę Riemanna za pomocą funkcji pierwotnej.}
Jeżeli ${f \colon [a,b] \to \mathbb{R}}$ jest funkcją ciągłą, natomiast ${F \colon [a,b] \to \mathbb{R}}$
jest jej dowolną funkcją pierwotną, to zachodzi równość:
\begin{equation*}
	\int_a^b f(x) \mathrm{d}x = [ F(x) ]_a^b = F(b) - F(a)
\end{equation*}

\begin{proof}
	Zdefiniujmy funkcję $F$ dla każdego ${x \in [a,b]}$:
	\begin{equation*}
		g(x) = \textstyle \int_a^x f(t) \mathrm{d}t
	\end{equation*}

	Skoro funkcja $f$ jest ciągła, to na mocy twierdzenia o funkcji
	górnej granicy całkowania funkcja $g$ jest różniczkowalna
	i zachodzi równość ${g^\prime(x)=f(x)}$ we wszystkich punktach ${x \in (a,b)}$.
	Oznacza to, że $g$ jest funkcją pierwotną funkcji $f$.
	Ponieważ każde dwie funkcje pierwotne danej funkcji różnią się o stałą,
	to dla pewnej liczby rzeczywistej C oraz dowolnego ${x \in [a,b]}$
	zachodzi równość
	\begin{equation}
		\label{eq.cal1}
		F(x)=g(x)+C
	\end{equation}
	Z definicji funkcji $g$ wynika, że
	\begin{equation}
		\label{eq.cal2}
		\textstyle \int_a^b f(t) \mathrm{d}t = g(b)
	\end{equation}
	A skoro $g(a) = \textstyle \int_a^a f(t) \mathrm{d}t=0$,
	to możemy kontynuować obliczenia, zapisując
	\begin{equation}
		\label{eq.cal3}
		g(b)=g(b)-g(a)=(g(b)+C)-(g(a)+C)=F(b)-F(a)
	\end{equation}
	gdzie ostatnia równość wynika z \eqref{eq.cal1}. Połączenie \eqref{eq.cal2} z \eqref{eq.cal3} implikuje żądany wzór i kończy dowód twierdzenia.
\end{proof}

\section{Co to jest całka niewłaściwa w sensie Riemanna?}
\subsection{Całka niewłaściwa Riemanna w przedziale $[a,b)$ lub $(a,b]$}
Niech ${f \colon [a,b) \to \mathbb{R}}$ będzie funkcją całkowalną w sensie Riemanna
	na każdym z przedziałów domkniętych ${[a,\beta]}$,
	gdzie $a < \beta < b$.
	Załóżmy, że funkcja $f$jest nieograniczona w pewnym lewostronnym sąsiedztwie punktu $b$.
	\begin{definition}
		\label{def.1}
		Całką niewłaściwą Riemanna funkcji $f$ nazywamy granicę:
		\begin{equation*}
			\lim_{\beta \to b^-} \int _a ^{\beta} f(x) \mathrm{d}x
			\textnormal{\enskip i oznaczamy ją symbolem \enskip}
			\int_a^b f(x) \mathrm{d}x
			\textnormal{.}
		\end{equation*}
	\end{definition}

	Jeżeli granica istnieje i jest skończona, to mówimy, że całka niewłaściwa ${\int_a^b f(x) \mathrm{d}x }$ jest zbieżna. Natomiast jeżeli granica ta nie istnieje lub jest niewłaściwa, to mówimy, że całka niewłaściwa ${\int_a^b f(x) \mathrm{d}x}$ jest rozbieżna.

W analogiczny sposób definiuje się całkę niewłaściwą Riemanna ${\int_a^b f(x) \mathrm{d}x}$ funkcji $f$ określonej na przedziale ${(a,b]}$, jak również pojęcia jej zbieżności i rozbieżności. Przyjmujemy wówczas, że
\begin{equation*}
	\int_a^b f(x) \mathrm{d}x \coloneqq \lim_{\alpha \to a^+} \int_{\alpha}^b f(x) \mathrm{d}x
\end{equation*}

\subsection{Całka niewłaściwa Riemanna w przedziale $(a,b)$}
Niech ${f \colon (a, b) \to \mathbb{R}}$ będzie funkcją całkowalną w sensie Riemanna
w każdym przedziale domkniętym $[\alpha,\beta]$, przy czym $a < \alpha < \beta < b$.
Załóżmy, że funkcja $f$ jest nieograniczona w pewnym prawostronnym sąsiedztwie punktu $a$ oraz
w pewnym lewostronnym sąsiedztwie punktu $b$.
\begin{definition}
	\label{def.2}
	Całkę niewłaściwą Riemanna funkcji $f$ w $(a,b)$ definiujemy jako
	\begin{equation*}
		\int_a^b f(x) \mathrm{d}x
		\coloneqq
		\int_a^c f(x) \mathrm{d}x + \int_c^b f(x) \mathrm{d}x
		\textnormal{,\enskip gdzie $c$ jest dowolnie wybranym punktem z $(a,b)$.}
	\end{equation*}
\end{definition}
Jeżeli obie całki w powyższej sumie są zbieżne, to mówimy,
że całka $\int_a^b f(x) \mathrm{d}x$ jest zbieżna.
Gdy któraś z tych całek nie istnieje lub jest rozbieżna, to mówimy,
że całka $\int_a^b f(x) \mathrm{d}x$ jest rozbieżna.

Jeżeli całka $\int_a^b f(x) \mathrm{d}x$ jest zbieżna, to jej wartość nie zależy od wyboru punktu $c$.

\subsection{Całka niewłaściwa Riemanna w przedziale
\texorpdfstring{$(-\infty,b]$}{(-inf,b]},
\texorpdfstring{$[a,\infty)$}{[a,+inf)} lub
	\texorpdfstring{$(-\infty,\infty)$}{(-inf,+inf)}
}
W powyższych definicjach \ref{def.1} i \ref{def.2} można podstawić $-\infty$ w miejsce $a$ i/lub $\infty$ w miejsce $b$:
\begin{itemize}
	\item
		$\int_a^{\infty} f(x) \mathrm{d}x
		\coloneqq
		\lim_{\beta \to \infty} \int_a^{\beta} f(x) \mathrm{d}x$
	\item
		$\int_{-\infty}^b f(x) \mathrm{d}x
		\coloneqq
		\lim_{\alpha \to -\infty} \int_{\alpha}^b f(x) \mathrm{d}x$
	\item
		$\int_{-\infty}^{\infty} f(x) \mathrm{d}x
		\coloneqq
		\int_{-\infty}^{c} f(x) \mathrm{d}x + \int_{c}^{\infty} f(x) \mathrm{d}x$ ($c \in \mathbb{R}$).
\end{itemize}

\section{Obliczyć całkę niewłaściwą funkcji
	\texorpdfstring{$ 1 / x^p $}{1/x\textasciicircum p}
	w przedziale
	\texorpdfstring{$ [ 1, \infty ) $}{[1,inf)}. %]
	Dla jakich wartości parametru $p$ całka ta jest zbieżna?
}
Przypadek $p=1$:
\begin{equation*}
	\int_1^\infty \frac{\mathrm{d}x}{x}
	= \lim_{\beta \to \infty} \left. \ln x \right|_1^{\beta}
	= \lim_{\beta \to \infty} (\ln \beta -\ln 1)
	= \infty
\end{equation*}

Przypadek $p\neq 1$:
\begin{equation*}
	\int_1^{\infty}\frac{\mathrm{d}x}{x^p}
	= \int_1^{\infty} x^{-p} \mathrm{d}x
	= \lim_{\beta \to \infty}\int_1^\beta x^{-p} \mathrm{d}x
	= \lim_{\beta \to \infty} \left. \frac{x^{-p+1}}{-p+1} \right|_1^\beta
	= \lim_{\beta \to \infty} \left. \frac{1}{(1-p)x^{p-1}} \right|_1^\beta
	= \frac{1}{1-p} \lim_{\beta \to \infty} \left( \frac{1}{\beta^{p-1}} - \frac{1}{1^{p-1}} \right)
\end{equation*}

Zauważmy, że
\begin{equation*}
	\lim_{\beta \to \infty} \frac{1}{\beta^{p-1}}
	= \left\{
		\begin{array}{ll}
			\infty & \text{ gdy } p-1< 0
			\\ 0 & \text{ gdy } p-1> 0
		\end{array}
	\right. %}
\end{equation*}

A zatem
\begin{equation*}
	\int_1^\infty \frac{\mathrm{d}x}{x^p}
	= \left\{
		\begin{array}{ll}
			\infty & \text{ gdy } p< 1
			\\ \frac{1}{(p-1)1^{p-1}} & \text{ gdy } p> 1
		\end{array}
	\right. %}
\end{equation*}

\textit{Reasumując:}
\begin{equation*}
	\int_1^\infty \frac{\mathrm{d}x}{x^p}
	= \left\{
		\begin{array}{lcl}
			\infty & \textnormal{ (jest rozbieżna) } & \text{ gdy } p\leq1
			\\ \frac{1}{(p-1)} & \textnormal{ (jest zbieżna) } & \text{ gdy } p>1
		\end{array}
	\right. %}
\end{equation*}

\section{Podać i udowodnić kryterium całkowe zbieżności szeregu.
	Zastosować to kryterium do szeregu harmonicznego z wykładnikiem
	\texorpdfstring{$ \alpha $}{a}.
}
\subsection{Teoria}
Niech $f \colon [n_0,\infty) \to \mathbb{R}$, $f(n)$ jest nierosnąca i nieujemna dla $n> n_0$. %]
[Analogicznie dla niemalejącej i niedodatniej.]
\begin{equation*}
	\textnormal{Wówczas: \enskip}
	\sum_{n={n_0}}^\infty f(n) \textnormal{ \enskip jest zbieżny}
	\iff
	\int_{n_0}^\infty f(n) \mathrm{d}n \textnormal{ \enskip jest zbieżna}
\end{equation*}
\begin{proof}%{\textit{(z wykładu)}}
	Weźmy $a_n \coloneqq \int_n^{n+1} f(x) \mathrm{d}x$ ($n \geq n_0$, $n \in \mathbb{N}$).
	Zauważmy, że $f(n)$ jest równe polu prostokąta o bokach $1$~i~$f(n)$.
	Mamy:
	$a_n \leq f(n) \leq a_{n-1}$
		% ${\int_n^{n+1} f(x) \mathrm{d}x \leq f(n) \leq \int_{n-1}^n f(x) \mathrm{d}x}$
	.
	\begin{multicols}{2}
		\centering
								% rysunki do dowodu
		\begin{tikzpicture}
				[
					scale=0.8,
					every circle/.style={radius=.05},
					sum/.style={pattern=north west lines,pattern color=yellow},
					declare function={
						f(\x) = {4.5 / \x};
					}
				]
			\draw [sum] (2,0) rectangle (1,{f(1)});
			\draw [sum] (3,0) rectangle (2,{f(2)});
			\draw [sum] (4,0) rectangle (3,{f(3)});
			\draw [sum] (5,0) rectangle (4,{f(4)});

				\fill [pattern=north east lines, pattern color=purple,domain=1:5,variable=\x]
				(1,0)
				-- plot (\x, {f(\x)})
				-- (5,0)
				-- cycle;

			\draw [domain=1:6] plot (\x, {f(\x)});
			\draw [fill] (1,{f(1)}) circle;
			\draw [thick,<->]
				(0,5) node [left] {$y$} --
				(0,0) --
				(6,0) node [below] {$x$};
			\draw [fill] (1,0) node [below] {$\phantom{-1} n_0 \phantom{-1}$} circle;
			\draw [fill] (3,0) node [below] {$n-1$} circle;
			\draw [fill] (4,0) node [below] {$\phantom{-1}n\phantom{-1}$} circle;
			\draw [fill] (5,0) node [below] {$n+1$} circle;
		\end{tikzpicture}
			% \caption{$\int_{n_0}^{n+1}f(x) \mathrm{d}x \leq \sum_{i=n_0}^nf(i)$}
		\begin{equation*}
			\int_{n_0}^{n+1}f(x) \mathrm{d}x \leq \sum_{i=n_0}^nf(i)
		\end{equation*}
		\columnbreak
		\vfill\null
		\begin{tikzpicture}
				[
					scale=0.8,
					every circle/.style={radius=.05},
					sum/.style={pattern=north west lines,pattern color=yellow},
					declare function={
						f(\x) = {4.5 / \x};
					}
				]
			\draw [sum] (1,0) rectangle (2,{f(2)});
			\draw [sum] (2,0) rectangle (3,{f(3)});
			\draw [sum] (3,0) rectangle (4,{f(4)});

				% \draw [pattern=crosshatch dots, pattern color=purple] (0,0) rectangle (1,{f(1)});
				\fill [pattern=north east lines, pattern color=purple,domain=1:4,variable=\x]
				(1,0)
				-- plot (\x, {f(\x)})
				-- (4,0)
				-- cycle;

			\draw [domain=1:6] plot (\x, {f(\x)});
			\draw [fill] (1,{f(1)}) circle;
			\draw [thick,<->]
				(0,5) node [left] {$y$} --
				(0,0) --
				(6,0) node [below] {$x$};
			\draw [fill] (1,0) node [below] {$\phantom{-1} n_0 \phantom{-1}$} circle;
			\draw [fill] (3,0) node [below] {$n-1$} circle;
			\draw [fill] (4,0) node [below] {$\phantom{-1} n \phantom{-1}$} circle;
			\draw [fill] (5,0) node [below] {$n+1$} circle;
		\end{tikzpicture}
			% \caption{$\sum_{i=n_0}^nf(i) \leq f(n_0) + \int_{n_0}^nf(x) \mathrm{d}x $}
		\begin{equation*}
			\begin{aligned}
				\sum_{i=n_0+1}^n f(i) & \leq \int_{n_0}^n f(x) \mathrm{d}x & \Big/ + f(n_0)
				\\\sum_{i=n_0}^n f(i) & \leq \int_{n_0}^nf(x) \mathrm{d}x + f(n_0) &
			\end{aligned}
		\end{equation*}
	\end{multicols}
	Stąd:
	\begin{equation*}
		\begin{CD}
			\displaystyle\int_{n_0}^{n+1}f(x) \mathrm{d}x @.
			\quad \leq \quad @.
			\displaystyle\sum_{i=n_0}^nf(i) @.
			\quad \leq \quad @.
			f(n_0) + @. \underbrace{\int_{n_0}^nf(x) \mathrm{d}x}
			\\ @VV{n \to \infty}V @. @VV{n \to \infty}V @. @. @VV{n \to \infty}V
			\\ \displaystyle\int_{n_0}^\infty f(x) \mathrm{d}x @.
			\quad \leq \quad @.
			\displaystyle\sum_{i=n_0}^\infty f(i) @.
			\leq @. f(n_0) +
			@. \displaystyle\int_{n_0}^\infty f(x) \mathrm{d}x
		\end{CD}
	\end{equation*}

	Jeżeli całka po lewej jest rozbieżna ($=\infty$) to szereg musi być rozbieżny.
	Jeśli szereg jest rozbieżny to całka po prawej musi być rozbieżna.
\end{proof}

\subsection{Przykład}
\begin{gather*}
	\alpha > 0 ,\quad \displaystyle\sum_{n=1}^\infty \frac{1}{n^\alpha} ,\quad f(x) = \frac{1}{x^\alpha}
	\\\int_1^\infty \frac{\mathrm{d}x}{x^\alpha}
	= \left. \frac{x^{1-\alpha}}{1-\alpha} \right|_1^\infty
	= \left\{
		\begin{array}{lr}
			\frac{1}{1-\alpha} & \alpha > 1
			\\\frac{\infty-1}{1-\alpha} & \alpha < 1
		\end{array}
	\right. %}
\end{gather*}
Szereg jest zbieżny $\iff \alpha > 1$

\section{Podać wzór na długość krzywej zadanej parametrycznie.}
Niech $-\infty < a < b < \infty$ oraz $f_1, f_2, \dots, f_n \colon [a,b] \to \mathbb{R}$ będą funkcjami ciągłymi klasy $C^1$. Krzywą nazywamy zbiór punktów ${\gamma = \{ (x_1, x_2, \dots, x_n) \in \mathbb{R}^n \colon x_1 = f_1 (t), x_2 = f_2 (t), \dots, x_n = f_n (t), t \in [a,b] \} }$.

Wówczas długość krzywej wyraża się wzorem:
\begin{equation*}
	| \gamma | = \int_a^b \sqrt{x_1^\prime(t)^2 + x_2^\prime(t)^2 + \dots + x_n^\prime(t)^2} \mathrm{d}t
\end{equation*}

W szczególności, dla $n=2$, $x_1=t$, $x_2=f(t)$:
\begin{equation*}
	| \gamma | = \int_a^b \sqrt{1 + f ^\prime (t) ^2} \mathrm{d}t
\end{equation*}

\section{Podać wzór na objętość i pole powierzchni bryły obrotowej.}
\begin{multicols}{2}
	$V$ - bryła powstała przez obrót wykresu $y=f(x)$ wokół osi Ox. $f \colon [a,b] \to [0,\infty)$. %]
	\begin{equation*}
		\textnormal{objętość: \enskip}
		| V | = \pi \int_a^b f(x)^2 \mathrm{d}x
	\end{equation*}
	\begin{equation*}
		\textnormal{pole} (V) = 2\pi \int_a^b \sqrt{1 + f^\prime(x)^2} \mathrm{d}x
	\end{equation*}
	\centering
	\def\svgwidth{20em}
	\input{obrot_bryly.pdf_tex}
\end{multicols}

\end{document}
