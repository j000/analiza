\documentclass{article}
% packages
% page geometry
\usepackage[
	a4paper,
	margin=1.5cm,
	includehead,
	includefoot
]{geometry}
% utf8 support
\usepackage[utf8]{inputenc}
% because...
\usepackage[T1]{fontenc}
% language support
\usepackage[polish]{babel}
% better math
\usepackage{amsmath}

% hyperlinks formatting
\usepackage[
	colorlinks,
	linkcolor=black,
	urlcolor=black,
	citecolor=black,
	plainpages=false,
	pdfpagelabels,
	backref,
	breaklinks
]{hyperref}
\makeatletter
\renewcommand{\Hy@numberline}[1]{#1. }
\makeatother

%custom headers and footers
\usepackage{fancyhdr}
\pagestyle{fancy}
\fancyhf{}
\renewcommand{\headrulewidth}{0pt}
\makeatletter
\rhead{\footnotesize \@title}
\cfoot{\footnotesize Page \thepage \hspace{1pt} of \pageref{LastPage}}
\makeatother

% for \LastPage
\usepackage{lastpage}

% dotted lines in ToC
\usepackage{tocloft} % for dots in ToC
\renewcommand{\cftsecleader}{\cftdotfill{\cftdotsep}}
\renewcommand{\cftsecaftersnum}{. }

% dots in bookmarks
\usepackage{bookmark}
\bookmarksetup{
	numbered,
	open,
}

% don't hypenate sections
\usepackage[raggedright]{titlesec}
% smaller font for title dots in section titles
\titleformat{\section}{\normalfont\fontsize{12}{15}\bfseries}{\thesection.}{1em}{}

% define title
\title{Zagadnienia z analizy}

\begin{document}

% bookmark title page
\makeatletter
\pdfbookmark[0]{\@title}{title}
\makeatother

% title page
\pagenumbering{gobble}
\begin{titlepage}
	\makeatletter
	\vfill
	\begin{center}
		\vspace*{0.3\textheight}
		{\LARGE
			\@title
		\par}
	\end{center}\par
	\vfill
	\@thanks
	\vfill
	\tableofcontents
	\vfill
	\makeatother
\end{titlepage}

\newpage

\linespread{1.3}
\pagenumbering{arabic}
\setcounter{page}{1}

\setcounter{section}{63}
\section{Podać i udowodnić zasadnicze twierdzenie rachunku całkowego,
które pozwala obliczyć całkę Reimanna za pomocą funkcji pierwotnej.}
Text


\section{Co to jest całka niewłaściwa w sensie Reimanna?}
Text


\section{Obliczyć całkę niewłaściwą funkcji
	\texorpdfstring{$ 1 / x^p $}{1/x\textasciicircum p}
	w przedziale
	\texorpdfstring{$ [ 1, \infty ) $}{[1,inf)}.
	Dla jakich wartości parametru p całka ta jest zbieżna?}
Text


\section{Podać i udowodnić kryterium całkowe zbieżności szeregu.
	Zastosować to kryterium do szeregu harmonicznego z wykładnikiem
	\texorpdfstring{$ \alpha $}{a}.}
Text


\section{Podać wzór na długość krzywej zadanej parametrycznie.}
Text


\section{Podać wzór na objętość i pole powierzchni bryły obrotowej.}
Text


\end{document}
